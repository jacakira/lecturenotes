\chapter[Foundations]{Foundations of Mathematics}

\section[Axiom of Choice]{The Axiom of Choice and Zorn's Lemma}

\marginpar{
  Lecture 1\\
  27 May 2024
}
The foundations of modern mathematics are it's axioms. The most widely used system of axioms in modern mathematics is Zermelo-Fraenkel Set Theory (\textit{ZFC}). We dedicate this section to the final axiom, the axiom of choice, which is of great importance in many areas of higher math.

We present the axiom of choice, and Zorn's Lemma without proof. See \cite{munkrestopology} for further details.

\begin{definition}
  Given a collection $\mathcal{A}$ of disjoint nonempty sets, there exists a set $C$ consisting of exactly one element from each element of $\mathcal{A}$; that is, a set $C$ such that $C$ is contained in the union of the elements of $\mathcal{A}$, and for each $A \in \mathcal{A}$, the set $C\cap A$ contains a single element.
\end{definition}

\begin{lemma}[Zorn's Lemma]
  Let $A$ be a set that is strictly partially ordered. If every simply ordered subset of $A$ has an upper bound in $A$, then $A$ has a maximal element.
\end{lemma}

This fairly innocuous axiom, and the succeeding lemma form the base of modern mathematics. We show one such example below.

\begin{theorem}
  Every vector space has a basis.
\end{theorem}

\begin{proof}
  This is a direct consequence of the Axiom of Choice.
  Consider the vector space $V$ over the field $F$ and the set
  \[
    \mathcal{A} = \{A \subset V \mid \text{All elements of } A \text{ are linearly independent in } V \}    
  \]

  We induce a partial order on this set: if $A_1, A_2 \in \mathcal{A}$ and $A_1 \subseteq A_2$ then $A_1 \leq A_2$.

  We now show that every subset of $\mathcal{A}$ has a maximal element. Suppose $\mathcal{A}_1$ is a partially ordered subset of $\mathcal{A}$. We show that
\begin{equation*}
  A_{\mathcal{A}_1} = \bigcup_{\mathcal{A}_1}A
\end{equation*}
is a maximal element. Clearly, $A_{\mathcal{A}_1}$ contains all elements of $\mathcal{A}_1$ and is thus larger than all elements of $\mathcal{A}$. We show now that $A_{\mathcal{A}_1}$ is linearly independent.

Suppose that $a_1, a_2, \cdots, a_n$ are an arbritrary set of elements of $A_{\mathcal{A}_1}$. As $\mathcal{A}$ is simply ordered, there must exist a set $A_{k} \in A_{\mathcal{A}_1}$ such that $a_1, a_2, \cdots, a_n \in A_{k}$. As $A_k \in \mathcal{A}$, $a_1, a_2, \cdots, a_n$ are linearly independent and thus $A_{\mathcal{A}_1}$ is linearly independent.
        
        We have shown that every simply ordered subset of $\mathcal{A}$ has an upper bound in $\mathcal{A}$. Thus, by Zorn's Lemma, $\mathcal{A}$ has a maximal element $M$. We now show that $M$ is a basis of $V$.

  Suppose for contradiction that there exists $x \in V$ such that $x \notin \langle M \rangle$. We now let $x_1, x_2, \cdots, x_n$ be a collection of vectors in $M$ and $a_1, \cdots,a_{n + 1} \in F$ such that:
  \[
    a_1x_1 + a_2x_2 + \cdots + a_nx_n - a_{n + 1}x = 0
  \]
  If $a_{n + 1}=0$ this would mean that $a_i = 0$ for all $1 \leq i \leq n$, contradicting the fact that $M$ is the maximal linearly independent member of $\mathcal{A}$. Thus,  $a_{n + 1}\neq 0$ and
  \[
    x = \sum_{i = 1}^n{\frac{a_i}{a_{n + 1}}x_i}    
  \]
  Therefore $x \in \mathrm{span}{M}$. This is a contradiction, and therefore $M$ is a basis for $V$.

\end{proof}


