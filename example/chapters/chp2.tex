\chapter[Group Theory]{An Introduction to Group Theory}
\section{Normal Subgroups and Quotient Groups}

\marginpar{
  Lecture 2\\
  28 May 2024
}
In this section we define a normal subgroup, which allows us to define quotient groups which are of great importance in further study of Groups.

\begin{definition}
  Let $G$ be a group and $H \leq G$. If $ghg^{-1} \in H$ for all $g \in G$, $h \in H$, then $H$ is a normal subgroup of $G$. We write this as $H \lhd G$.
\end{definition}

\begin{example}
  Let $G$ be an abelian group and $H \leq G$. Then as $ghg^{-1}=gg^{-1}h=1_Gh=h \in H$, we can conclude that $H \lhd G$.  
\end{example}

\begin{theorem}
  Let $G_1, G_2$ be groups and $\phi : G_1 \rightarrow G_2$ be a homomorphism. Then, $\text{Ker}(\phi)$ is a normal subgroup.
\end{theorem}
\begin{proof}
  Let \( g \in G \) and \( h \in \ker(\phi) \). Then,
  \[
    \phi(ghg^{-1}) = \phi(g)\phi(h)\phi(g)^{-1} = \phi(g)1_G\phi(g)^{-1} = \phi(g)\phi(g)^{-1} = 1_G.
  \]
  Therefore, \( ghg^{-1} \in \ker(\phi) \).
\end{proof}

We note that with this definition of a normal subgroup, it is natural to define the following equality between cosets:
\[
  (gN)(hN)\overset{\text{def}}{=}ghN
\]
With this definition, the quotient $G/N$ forms a group.

\section{The First Isomorphism Theorem}

In this section we introduce the first isomorphism theorem and its corollaries.

\begin{theorem}[First Isomorphism Theorem]
  \begin{flushleft}
    \begin{minipage}{0.6\textwidth}
      \label{firstiso}
      Let $\phi: G \rightarrow H$ be a homomorphism of groups. If we let $\pi : G \rightarrow G/\text{Ker}(\phi)$ be the canonical projection map, there exists a unique isomorphism $\psi: G/\text{Ker}(\phi) \rightarrow H$.
    \end{minipage}%
    \hfill
    \begin{minipage}{0.35\textwidth}
      \begin{tikzcd}
        G \arrow[r, "\phi"] \arrow[d, "\pi"'] & H \\
        G/\text{Ker}(\phi) \arrow[ur, dashed, "\psi"'] &
      \end{tikzcd}
    \end{minipage}
  \end{flushleft}
\end{theorem}

\begin{corollary}[Rank-Nullity Theorem]
  Let $V$ and $W$ be vector spaces and $T:V\rightarrow W$ be a linear map. Then,
  \begin{equation*}
    \text{dim}V = \text{dim}\text{Im} T + \text{dim}\text{Ker}T
  \end{equation*}
\end{corollary}

\begin{proof}
  We note that $(V, \oplus)$ forms an abelian group. Furthermore, we note that the linear map $T$ is a homomorphism. Therefore, by Theorem \ref{firstiso}, we have 
  \begin{align*}
    V/\text{Ker}T &\cong \text{Im} \\
    \text{dim}(V/\text{Ker}T) &= \text{dim}\text{Im}T\\
    \text{dim}V-\text{dim}\text{Ker}T&=\text{dim}\text{Im}T
  \end{align*}
  The result follows immediately.
\end{proof}

